


\documentclass[12pt, a4paper]{exam}
\usepackage{graphicx}
 \usepackage[left=0.8in, top=0.7in, bottom=1in, total={6.2in,8in}]{geometry}
\usepackage[normalem]{ulem}
\usepackage{mathptmx}
\usepackage{enumitem}
\usepackage[english]{babel}
% \usepackage{fancyhdr}
\usepackage{lastpage}
\usepackage{mathtools}

%\renewcommand{\questionlabel}{\textbf{Q ~\thequestion.}}


% \fancyhf{}
\rfoot{Page \textbf{\thepage} \hspace{1pt} of \textbf{\pageref{LastPage}}}

\renewcommand\ULthickness{1.0pt}   %%---> For changing thickness of underline
\setlength\ULdepth{1.3ex}%\maxdimen ---> For changing depth of underline

%% Define a new command as Steven B. Segletes suggested:
%% Use the length \rlength to have a default width.
\newlength{\rlength}\setlength{\rlength}{\textwidth}
\newcommand{\ruletext}[2][\rlength]{%
  \noindent%
  \parbox{#1}{%
    \noindent\hrulefill\raisebox{-.3\ht\strutbox}{#2}\hrulefill\par}}


\marksnotpoints
%\bracketedpoints
\pointsinrightmargin
\qformat{\textbf{Q \thequestion \quad \thequestiontitle\hfill\thepoints}}

\begin{document}

	%\thispagestyle{empty}
	\noindent
	\begin{minipage}[l]{0.1\textwidth}
		\noindent
		\includegraphics[width=1.5\textwidth]{RIU logo.png}
	\end{minipage}
	\hfill
	\begin{minipage}[c]{1\textwidth}
		\begin{center}
			\large	\textbf{Department of Electrical Engineering} \par
			\large	\textbf{Faculty of Engineering \& Applied Sciences		}	\par
		    Riphah International University	\par
		    \textbf{Midterm Examinations, Summer 2024 Semester} \par 
		    \textbf{\small B.Sc. Electrical Engineering Program} 
		\end{center}
	\end{minipage}
\par
	\vspace{0.2in}

	\noindent
	\textbf{SAP ID :} \hspace{0.5in} \rule{3cm}{0.15mm} \hspace{0.43in}  \textbf{Subject Name :} \hspace{0.5in} Electromagnetic Field \par
	\vspace{0.2in}
	\noindent
	\textbf{Marks :} \hspace{1in} 30 \hspace{1in} \textbf{Time Allowed :} \hspace{0.5in} 120 minutes \par

\vspace{0.2in}
\noindent
\begin{footnotesize}
	\textbf{Instructions:
		\begin{enumerate}[nosep, leftmargin=16pt]
			\item All the parts (if any) of each question must be attempted at one place instead of at different places.
			\item Write Q. No. in the answer book in accordance with Q. No. in the Q. Paper.
			\item Extra attempt of any question or any part of the attempted question will not be considered.
		\end{enumerate}}
\end{footnotesize}
\noindent\rule{\textwidth}{1pt}


\begin{questions}
%	\bracketedpoints
	\pointsinrightmargin
	\titledquestion{CLO1}[10]
	You are working as an engineer in a telecommunications company, tasked with positioning a new antenna. The antenna's current location is given in Cartesian coordinates as $P(3, 4, 5)$. To accurately align the antenna with existing infrastructure, you need to convert this location into both cylindrical and spherical coordinates. 
	\begin{parts}
		\part Convert the given point $ P(3, 4, 5) $ in Cartesian coordinates to cylindrical coordinates
		\part Convert the same point to spherical coordinates using the equations.
		\part Individually plot the point in the three coordinate systems.
	\end{parts}
	
		
	\vspace{0.2in}
	
	\titledquestion{CLO1}[10]
	You are an electromagnetic field engineer assigned to calculate the interaction between two charged particles in free space for a high-precision experimental setup. Charge $Q_A = -20 \, \mu C$ is located at $A(-6, 4, 7)$, and charge $Q_B = 50 \, \mu C$ is at $B(5, 8, -2)$. All distances are given in meters.
	\begin{parts}
		\part Calculate the distance $R_{AB}$ between points $A$ and $B$
		\part Determine the unit vector $\hat{R}_{AB}$ pointing from $A$ to $B$.
		\part Compute the vector force exerted on $Q_A$ by $Q_B$ using the permittivity $\epsilon_0 = \frac{10^{-9}}{36\pi} \, \text{F/m}$.
		\part Recalculate the vector force with the permittivity $\epsilon_0 = 8.854 \times 10^{-12} \, \text{F/m}$.
	\end{parts}
	
	\vspace{0.2in}
	
	\titledquestion{CLO2}[10]
	You are tasked with analyzing the electric field properties in a region of free space where the electric flux density is given by $\mathbf{D} = 0.3r^2 \hat{a}_r \, \text{nC/m}^2$. You need to perform the following calculations for a report on the electric field distribution and charge within a spherical region.
	\begin{parts}
		\part Find the electric field $\mathbf{E}$ at point $P(r = 2, \theta = 25^\circ, \phi = 90^\circ)$.
		\part Determine the total charge within a sphere of radius $r = 3$.
		\part Calculate the total electric flux leaving a sphere of radius $r = 4$.
	\end{parts}	
\end{questions}



\vspace{0.2in}
\ruletext{\large \bfseries End of Paper}
\end{document}